% arara: pdflatex: {files: [MathSACpr2014]}
% !arara: indent: {overwrite: yes}
\chapter{Enrollment summaries (by term and campus)}\label{sec:app:enrollment}

\begin{figure}[!htb]
	\centering
	% arara: pdflatex
% !arara: indent: {overwrite: yes}
\documentclass{standalone}
% Caption: Enrollment in Developmental MTH by TERM
% 2008-2012

\usepackage{pgfplots,pgfplotstable}
\pgfplotsset{compat=newest}

\begin{document}

\pgfplotstableread{
	Year    Summer  Fall    Winter  Spring
	2008  1964    6322    6085    5962
	2009  2586    7657    8008    7565
	2010  3330    8289    8104    7683
	2011  3284    9148    8944    8269
	2012  3097    8946    8467    7531
	}\mydata

\begin{tikzpicture}
	\begin{axis}[
			%ybar,
			symbolic x coords={2008, 2009, 2010, 2011, 2012},
			xtick=data,
			minor ytick={1000,2000,...,10000},
			xticklabels = {2008/09,2009/10, 2010/11, 2011/12, 2012/13},
			enlarge x limits,
			scale only axis,       
			xticklabels = {2008/09,2009/10, 2010/11, 2011/12, 2012/13},
			grid = both,
			ymin=0,ymax=10000,
			scaled ticks=false, 
			tick label style={/pgf/number format/fixed},
			legend pos= outer north east,
			width=0.6\textwidth,
			x tick label style={rotate=25},
		]
		\addplot table[x=Year,y=Fall]{\mydata};
		\addplot table[x=Year,y=Winter]{\mydata};
		\addplot table[x=Year,y=Spring]{\mydata};
		\addplot table[x=Year,y=Summer]{\mydata};
		%\addplot table[x=Year,y expr=\thisrow{Summer}+\thisrow{Fall}+\thisrow{Winter}+\thisrow{Spring}]{\mydata};
		\legend{Fall, Winter, Spring,Summer, Total}
	\end{axis}
\end{tikzpicture}
\end{document}

	\caption{Enrollment in Developmental MTH by Term}
\end{figure}
\begin{figure}[!htb]
	\centering
	% arara: pdflatex
% !arara: indent: {overwrite: yes}
\documentclass{standalone}
% Caption: Developmental Mathematics Enrollment by CAMPUS
% 2008-2012

\usepackage{pgfplots,pgfplotstable}
\pgfplotsset{compat=newest}

\begin{document}

\pgfplotstableread{
    Year SY  CA  RC  ELC
    2008  6764    4159    6625    2785
    2009  8155    5745    8033    3883
    2010  8847    5963    8192    4404
    2011  9682    6585    8669    4709
    2012  8840    5887    8454    4860
}\mydata

\begin{tikzpicture}
	\begin{axis}[
			%ybar,
			symbolic x coords={2008, 2009, 2010, 2011, 2012},
			xtick=data,
			minor ytick={1000,2000,...,10000},
			enlarge x limits,
			scale only axis,       
			xticklabels = {2008/09,2009/10, 2010/11, 2011/12, 2012/13},
			grid = both,
			ymin=0,ymax=10000,
			scaled ticks=false, 
			tick label style={/pgf/number format/fixed},
			legend pos= south east,
		]
		\addplot table[x=Year,y=SY]{\mydata};
		\addplot table[x=Year,y=CA]{\mydata};
		\addplot table[x=Year,y=RC]{\mydata};
		\addplot table[x=Year,y=ELC]{\mydata};
		%\addplot table[x=Year,y expr=\thisrow{SY}+\thisrow{CA}+\thisrow{RC}+\thisrow{ELC}]{\mydata};
		\legend{SY, CA, RC, ELC, Total}
	\end{axis}
\end{tikzpicture}
\end{document}

	\caption{Enrollment by campus and year, College Wide, Developmental Math}
\end{figure}
\begin{figure}[!htb]
	\centering
	% arara: pdflatex
% !arara: indent: {overwrite: yes}
\documentclass{standalone}
% Caption: Lower Division College Transfer enrollment by term
% 2008-2012

\usepackage{pgfplots,pgfplotstable}
\pgfplotsset{compat=newest}

\begin{document}

\pgfplotstableread{
Year	Summer	Fall	Winter	Spring
 2008	1220	2425	2532	2627
 2009	1484	3068	3226	3250
 2010	2037	3450	3538	3494
 2011	1911	5446	5238	5202
 2012	2599	5513	5242	5189
	}\mydata

\begin{tikzpicture}
	\begin{axis}[
			%ybar,
			symbolic x coords={2008, 2009, 2010, 2011, 2012},
			xtick=data,
			minor ytick={1000,2000,...,10000},
			enlarge x limits,
			scale only axis,       
			xticklabels = {2008/09,2009/10, 2010/11, 2011/12, 2012/13},
			grid = both,
			ymin=0,ymax=10000,
			scaled ticks=false, 
			tick label style={/pgf/number format/fixed},
			legend pos= outer north east,
			width=0.6\textwidth,
			x tick label style={rotate=25},
		]
		\addplot table[x=Year,y=Fall]{\mydata};
		\addplot table[x=Year,y=Winter]{\mydata};
		\addplot table[x=Year,y=Spring]{\mydata};
		\addplot table[x=Year,y=Summer]{\mydata};
		%\addplot table[x=Year,y expr=\thisrow{Summer}+\thisrow{Fall}+\thisrow{Winter}+\thisrow{Spring}]{\mydata};
		\legend{Fall, Winter, Spring, Summer, Total}
	\end{axis}
\end{tikzpicture}
\end{document}

	\caption{Enrollment in LDC by term}
\end{figure}
\begin{figure}[!htb]
	\centering
	% arara: pdflatex
% !arara: indent: {overwrite: yes}
\documentclass{standalone}
% Caption: Enrollment in LDC MTH by CAMPUS
% 2008-2012

\usepackage{pgfplots,pgfplotstable}
\pgfplotsset{compat=newest}

\begin{document}

\pgfplotstableread{
Year	SY	CA	RC	ELC
2008	4096	1497	2920	291
2009	4883	2036	3625	484
2010	5405	2042	4451	621
2011	7173	3155	6262	1207
2012	7297	3435	6424	1387
}\mydata

\begin{tikzpicture}
	\begin{axis}[
			%ybar,
			symbolic x coords={2008, 2009, 2010, 2011, 2012},
			xtick=data,
			minor ytick={1000,2000,...,10000},
			enlarge x limits,
			scale only axis,       
			grid = both,
			ymin=0,ymax=10000,
			scaled ticks=false, 
			tick label style={/pgf/number format/fixed},
			legend pos=outer north east,
		]
		\addplot table[x=Year,y=SY]{\mydata};
		\addplot table[x=Year,y=CA]{\mydata};
		\addplot table[x=Year,y=RC]{\mydata};
		\addplot table[x=Year,y=ELC]{\mydata};
		%\addplot table[x=Year,y expr=\thisrow{SY}+\thisrow{CA}+\thisrow{RC}+\thisrow{ELC}]{\mydata};
		\legend{SY, CA, RC, ELC, Total}
	\end{axis}
\end{tikzpicture}
\end{document}

	\caption{Enrollment in LDC MTH by campus}
\end{figure}
\begin{figure}[!htb]
	\centering
	% arara: pdflatex
% !arara: indent: {overwrite: yes}
\documentclass{standalone}
% Caption: Combined Math enrollment by term and year
% 2008-2012

\usepackage{pgfplots, pgfplotstable}
\pgfplotsset{compat=newest}

\begin{document}

\pgfplotstableread{
	Year    Summer  Fall    Winter  Spring
	2008  3184    8747    8617    8589
	2009  4070    10725   11234   10815
	2010  5367    11739   11642   11177
	2011  5195    14594   14182   13471
	2012  5696    14459   13709   12720
	}\mydata

\begin{tikzpicture}
	\begin{axis}[
			%ybar stacked,
			symbolic x coords={2008, 2009, 2010, 2011, 2012},
			xtick=data,
			%minor ytick={5000,15000,25000,35000,45000},
			enlarge x limits,
			scale only axis,       
			xticklabels = {2008/09,2009/10, 2010/11, 2011/12, 2012/13},
			grid = both,
			ymin=0,ymax=50000,
			scaled ticks=false, 
			tick label style={/pgf/number format/fixed},
			legend pos= outer north east,
			width=0.6\textwidth,
			x tick label style={rotate=25},
		]
		\addplot table[x=Year,y=Fall]{\mydata};
		\addplot table[x=Year,y=Winter]{\mydata};
		\addplot table[x=Year,y=Spring]{\mydata};
		\addplot table[x=Year,y=Summer]{\mydata};
		\addplot table[x=Year,y expr=\thisrow{Summer}+\thisrow{Fall}+\thisrow{Winter}+\thisrow{Spring}]{\mydata};

		\legend{Fall, Winter, Spring, Summer, Total}
	\end{axis}
\end{tikzpicture}
\end{document}

	\caption{Combined Math enrollment by term and year}
\end{figure}
\begin{figure}[!htb]
	\centering
	% arara: pdflatex
% !arara: indent: {overwrite: yes}
\documentclass{standalone}
% Caption: Combined Math enrollment by CAMPUS
% 2008-2012

\usepackage{pgfplots,pgfplotstable}
\pgfplotsset{compat=newest}

\begin{document}

\pgfplotstableread{
Year	SY		CA		RC		ELC		
2008	10860	5656		9545		3076	
2009	13038	7781		11658	4367
2010	14252	8005		12643	5025
2011	16855	9740		14931	5916
2012	16137	9322		14878	6247
}\mydata

\begin{tikzpicture}
	\begin{axis}[
			%ybar,
			symbolic x coords={2008, 2009, 2010, 2011, 2012},
			xtick=data,
			minor ytick={1000,2000,3000,4000,6000,7000,8000,9000,11000,12000,13000,14000,16000,17000},
			enlarge x limits,
			xticklabels = {2008/09,2009/10, 2010/11, 2011/12, 2012/13},
			scale only axis,       
			grid = both,
			ymin=0,ymax=18000,
			scaled ticks=false, 
			tick label style={/pgf/number format/fixed},
			legend pos=outer north east,
		]
		\addplot table[x=Year,y=SY]{\mydata};
		\addplot table[x=Year,y=CA]{\mydata};
		\addplot table[x=Year,y=RC]{\mydata};
		\addplot table[x=Year,y=ELC]{\mydata};
		%\addplot table[x=Year,y expr=\thisrow{SY}+\thisrow{CA}+\thisrow{RC}+\thisrow{ELC}]{\mydata};
		\legend{SY, CA, RC, ELC, Total}
	\end{axis}
\end{tikzpicture}
\end{document}

	\caption{Enrollment trends by campus (combined MTH)}
\end{figure}
\begin{figure}[!htb]
	\centering
	% arara: pdflatex
% !arara: indent: {overwrite: yes}
\documentclass{standalone}
% Caption: Ratio of developmental MTH enrollment to LDC MTH enrollment
% 2008-2012

\usepackage{pgfplots}
\pgfplotsset{compat=newest}

\begin{document}

\pgfplotstableread{
Year    Ratio
2008    2.3095184007
2009    2.3409503083
2010    2.1891524882
2011    1.6657301792
2012    1.512214852
	}\mydata

\begin{tikzpicture}
	\begin{axis}[
			symbolic x coords={2008, 2009, 2010, 2011, 2012},
			xtick=data,
			enlarge x limits,
			scale only axis,       
			grid = both,
			ymin=0,ymax=3,
			scaled ticks=false, 
			width=\textwidth,
		]
		\addplot table[x=Year,y=Ratio]{\mydata};
	\end{axis}
\end{tikzpicture}
\end{document}

	\caption{Ratio of Developmental MTH enrollment to LDC MTH enrollment}
\end{figure}
\begin{figure}[!htb]
	\centering
	% arara: pdflatex
% !arara: indent: {overwrite: yes}
\documentclass{standalone}
% Caption: Success rates by year and term (Success rates decline)
% 2008-2012

\usepackage{pgfplots,pgfplotstable}
\pgfplotsset{compat=newest}

\begin{document}

\pgfplotstableread{
	Year	Fall	Winter	Spring	Summer
	2008	0.6763020184	0.6954089506	0.6728676034	0.7086156825
	2009	0.6936344969	0.7029561672	0.6449203187	0.6685520362
	2009	0.6936344969	0.7029561672	0.6449203187	0.6685520362
	2010	0.6589605735	0.6775119617	0.6449203187	0.6820288363
	2011	0.6439919445	0.6241371632	0.5989427718	0.6448675497
	2012	0.6174436761	0.6360416205	0.6098600122	0.6244411326
	}\mydata

\begin{tikzpicture}
	\begin{axis}[
			%ybar stacked,
			axis y discontinuity=crunch,
			symbolic x coords={2008, 2009, 2010, 2011, 2012},
			xtick=data,
			enlarge x limits,
			scale only axis,       
			xticklabels = {2008/09,2009/10, 2010/11, 2011/12, 2012/13},
			grid = both,
			ymin=0.56,ymax=0.75,
			yticklabel=,
			scaled ticks=false, 
			tick label style={/pgf/number format/fixed},
			legend pos= outer north east,
			width=0.6\textwidth,
			x tick label style={rotate=25},
		]
		\addplot table[x=Year,y=Fall]{\mydata};
		\addplot table[x=Year,y=Winter]{\mydata};
		\addplot table[x=Year,y=Spring]{\mydata};
		\addplot table[x=Year,y=Summer]{\mydata};
		\legend{Fall, Winter, Spring, Summer}
	\end{axis}
\end{tikzpicture}
\end{document}

	\caption{Success rates by year and term}
\end{figure}

\fixthis{there's a table here for success rates- does it need to be here? it is commented}
%\begin{table}[!h]
%	\centering
%	\caption{Success rates by term and year}
%	\begin{tabular}{l*{6}{c}}
%		\toprule
%		& \multicolumn{6}{c}{Success rates}\\
%		\cmidrule{2-7}
%		       & AY2008  & AY2009  & AY2010  & AY2011  & AY2012  & AY2013  \\
%		\cmidrule{2-7}
%		Fall   & 69.85\% & 67.63\% & 69.36\% & 65.90\% & 64.40\% & 61.74\% \\ 
%		Winter & 69.47\% & 69.54\% & 70.30\% & 67.75\% & 62.41\% & 63.60\% \\
%		Spring & 67.33\% & 67.29\% & 64.49\% & 64.49\% & 59.89\% & 60.99\% \\
%		Summer & 72.65\% & 70.86\% & 66.86\% & 68.20\% & 64.49\% & 62.44\% \\
%		\bottomrule
%	\end{tabular}
%\end{table}
