% arara: pdflatex: {files: [MathSACpr2014]}
\chapter{Needs of Students and the Community}
\section{How is instruction informed by student demographics?}
In order to answer this question, we decided that we needed a definition of demographics beyond the normal categories that are provided by the college. We came up with age, sex, gender, race, creed, sexual orientation, learning ability, educational background, and socio-economic status.

Our instruction is informed by these student demographics in a variety of ways.

\subsection{Social Justice (Addresses Socio-Economic Status, Race, Gender)}
The Math SAC has a social justice workgroup that was formed in 2012.  Their objectives are to explore and discuss issues relating to diversity within the mathematics classroom as well as to create projects, activities, and other course content related to issues surrounding social and environmental justice.

Examples of these projects and exercises include: a fine in Yonkers, NY related to segregation
for MTH 111; racial profiling in traffic stops; gentrification in Portland; and the Deepwater Horizon Oil Spill. Many of these projects were adapted to fit various mathematical levels from MTH 20 to MTH 252. Problems were also generated for MTH 243, using gun violence and international prison data. For more detail, see \vref{app:sec:socialJustic}.

\subsection{Individual Faculty Awareness}
A recent survey of MTH faculty asked if they had ever modified instruction to meet our diversity goals. The survey used our previous program review's definition of diversity, which was:
\begin{quote}
  We will enrich the educational experience by committing to the development of diversity in our student body, faculty and staff.
\end{quote}
Here are some highlights and themes from the survey responses. One faculty member mentioned that 
\begin{quote}
  I have been learning about Complex Instruction, which has helped me attend to status in my classroom. Who has high status and who has low status? Complex Instruction (CI) provides opportunities to highlight the diversity of ways to be smart in a mathematics classroom\ldots so that all students can participate equally in the classroom activity.
\end{quote}
Another SAC member is dedicated to educating herself in the classroom, 
\begin{quote}
  If there is a cultural barrier, my awareness and appreciation of diversity enables me to want to learn about the unfamiliar, and educate about my own. My immense experience working in diverse settings with unique individuals constantly increases my awareness of what I can do to make someone fell comfortable and what I need to do to accept individuality without enforcing conformity.
\end{quote}
Many of our faculty commented on the use of group work as a way to expose students to diverse ideas and culture.  They also indicated that they tried to be culturally aware when writing application problems by choosing different names, genders and roles for the characters in their problems. The `Rule of Four' (functions and relations should be presented numerically, graphically, verbally and symbolically) is incorporated into most of our CCOGs. The rule of four recognizes and highlights the different ways people prefer to learn mathematics.
\subsection{Educational Cost (Addresses Socio-Economic Status)}
Our SAC is aware of the cost of course materials and considers the socio-economic status of our student population when selecting texts. 

The SAC has a long-standing policy to require the same textbook for all sections of a course. Since PCC has such a large student body and we offer many math classes, this policy has enabled the math SAC to negotiate wholesale prices of textbooks (particularly custom editions) with publishers.  This saves students money when taking a sequence course (for example MTH 60/65) and allows students to sell back their book to the bookstore.

Publishers can create a custom edition from an existing textbook by removing material (e.g., chapters) or adding material (e.g., supplemental materials); the publisher labels the textbook `A Custom Edition for Portland Community College', and thus restricts its resale value, as it can only be used at PCC; this benefits the publisher and enables them to reduce the price to PCC.  

Math SAC subcommittee have successfully implemented this idea with the textbooks for almost all of the mathematics classes taught at PCC: MTH 20, the MTH 60/61/62/63/65 sequences, MTH 95, MTH 111, MTH 112, MTH 105, MTH 243/244 sequence, and the MTH 251/252/253/254 sequence.

In addition to using custom editions uniformly across the district, we have a group that is investigating an in-house Pre-Calculus text to reduce dependency on publishers. The group is inactive at this point because they have been unable to secure funding or release time \fixthis{(REF other part of the document that describes the project- Hughes, Jordan, Cary, Simonds).}

We actively pursue free and open source products such as WeBWorK-- the only fully accessible online homework system \fixthis{REF WeBWorK section}. This meets our goal of providing low cost curricular materials and also supports student access. The University of Oregon has generously hosted several WeBWorK courses for PCC over the past few years.  Disability Services has provided strong support for WeBWorK, and we were able to procure our own WeBWorK server at PCC in the Fall of 2013.

\subsection{Educational Background}
We have several projects and classes in place to address our students' different educational backgrounds. \fixthis{reference section 3E}

\paragraph{ Study skills}
The Study Skills program was created to address the different educational backgrounds of our students, particularly those students who have underdeveloped study skills. This program consists of seven topics all relating to study skills specific to mathematics: how learning math is different, resources available for help at PCC, time management, listening and note-taking skills, why and how to do homework, test taking strategies, and how to overcome math and test anxiety.  Each lesson is broken up into three parts: a short video to be watched by students outside of class, a student worksheet to be completed in conjunction with the video, and an in-class discussion lead by the instructor. 

\paragraph{AMP}
AMP addresses differences in educational backgrounds. It allows students who have previous exposure to the material to attempt to move to a higher level class (see \vref{app:sec:ampdata}).

 Even with the above mentioned programs, we feel that the college could do a lot more when it comes to placing students into classes appropriately and orienting them to the demands of college. We would like to see more wrap-around services for students in developmental classes. These services would ideally begin before the students steps into the classroom. We suggest the adoption of a placement test that measures study skills, motivation and academic preparedness. 

 We recommend that students who are not academically prepared be required to take a study skills course. We would like to see more math-specific advisors and have enough advisors so that it is feasible for a student to see an advisor every term.  We would also like to see the tutoring center open during the first week of the term. In our experience, students who are behind during the first week have a hard time catching up.

 \subsection{Data Trends}
 Despite the above mentioned efforts to have instruction informed by our student demographics, we have still found that there is an achievement gap when it comes to minority and underrepresented populations. We have displayed data for five years in \vref{app:sec:demographicdata}; see \crefrange{app:tab:demographic2008-2009}{app:tab:demographic2012-2013}. PCC has undergone vast enrollment changes over the last five years since our previous program review; here are the trends that we observed for this time period:
 \begin{itemize}
   \item The percentage of both White and Asian students increases as students progress through the sequence of MTH classes. There appears to be a modest increase in diversity levels in MTH 251-254 over the last 5 years, but this may be due to more students identifying as Multiracial.
   \item There is slight increase in diversity since AY 2008 (the percentage of students who identify as white has decreased in most of our courses).  
   \item There is a shockingly high numbers of students aged 19 or less who place into MTH 20. Since many of these students should have been exposed to the material recently, we need to further examine both the placement exam and our communication with high schools. For example, are high schools allowing students to use calculators too freely? Are students who are otherwise proficient at algebra placing low due to not understanding fractions? This is something that needs further investigating. If we could decrease the number of young students placing into MTH 20, we might be able to shorten their path to a degree.
   \item The percentage of students aged 50+ decreases through the DE sequence. We suggest intentionally creating support systems for 50+ students, particularly in MTH 60. These students likely have been out of the educational system the longest, so they face different challenges than their younger peers.
   \item There is a large decrease in the percentage of black students from MTH 20 to MTH 60. Not only is there a percentage decrease, there is also a decrease in the total number of black students in MTH 60 compared to MTH 20. This indicates that MTH 20 is likely a significant barrier to some minority students or that minority students place into MTH20 at a disproportionately high rate. Although this is relatively consistent with national data, we would like the administration to continue to support programs like Passages and other measures to increase success rates of minority students. In addition, a more diverse faculty might help with retention and passing rates.
   \item The pass rates for black students are noticeably lower each year and in each course. We suggest intentionally creating support systems for black students studying mathematics.
   \item Females consistently pass MTH 20/60/65/95 at higher rates than males, but a smaller proportion of females enter MTH 112. We suggest identifying barriers to females continuing on to MTH 112 and related careers. However, we also realize that there are larger cultural and societal trends at play here.
   \item Female students are underrepresented in MTH 112, and 251-254 as noted above. However, it appears that many female students take a statistics route instead of a calculus route.
   \item The percentage of men passing MTH 20 is lower than that of female students. In addition, it appears that the percentage of males enrolling in MTH 20 is increasing (perhaps due to the economic downturn). This is consistent with data at the secondary level. Since MTH 20 is pre-algebra, some of this may be due to prior educational experiences and students attitudes of their ability.
 \end{itemize}

 \section{Have there been any notable changes in instruction due to changes in demographics since the last review?}
