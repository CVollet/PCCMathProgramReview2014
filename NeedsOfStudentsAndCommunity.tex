% arara: pdflatex: {files: [MathSACpr2014]}
% !arara: indent: {overwrite: yes}
\chapter{Needs of Students and the Community}
\epigraph{Everyone can rise above their circumstances and achieve success if they are dedicated to and passionate about what they do.}{Nelson Mandela}

\section[Student demographics and instruction]{How is instruction informed by student demographics?}
In order to answer this question, we decided that we needed demographic
categories beyond the normal categories that are provided by the college. We
decided to use age, sex, gender, race, creed, sexual orientation, learning
ability, educational background, and socio-economic status. Our instruction is
informed by these student demographics in a variety of ways.
\label{needs:sec:definitiondiversity}



\subsection[Social justice workgroup]{Social justice (addresses socio-economic status, race, gender)}
The Math SAC has a Social Justice Workgroup that was formed in 2012 (detailed
on \cpageref{cur:sub:socialJustic}).  Their
objectives are to explore and discuss issues relating to diversity within the
mathematics classroom as well as to create projects, activities, and other
course content related to issues surrounding social and environmental justice.

For example, in MTH 111 topics include: a fine in Yonkers, NY related
to segregation; racial profiling in traffic stops; gentrification
in Portland; and the Deepwater Horizon Oil Spill. Many of these projects were
adapted to fit various mathematical levels from MTH 20 to MTH 252. Problems
were also generated for MTH 243, using gun violence and international prison
data. Some samples of their work are given in \vref{app:sec:socialJustic}.

\subsection{Individual faculty awareness}
A recent survey of MTH faculty asked if they had ever modified instruction to
meet our diversity goals. The survey used our previous program review's
diversity statement (Goal 3 on page 4 of \cite{mathprogramreview2003}) as a point of reference:
\begin{quote}
	We will enrich the educational experience by committing to the development
	of diversity in our student body, faculty and staff.
\end{quote}
Here are some highlights and themes from the survey responses. One faculty
member reports
\begin{quote}
	I have been learning about Complex Instruction, which has helped me attend
	to status in my classroom. Who has high status and who has low status?
	Complex Instruction (CI) provides opportunities to highlight the diversity
	of ways to be smart in a mathematics classroom\ldots so that all students
	can participate equally in the classroom activity.
\end{quote}
Another SAC member is dedicated to educating herself in the classroom, and
reports
\begin{quote}
	If there is a cultural barrier, my awareness and appreciation of diversity
	enables me to want to learn about the unfamiliar, and educate about my own.
	My immense experience working in diverse settings with unique individuals
	constantly increases my awareness of what I can do to make someone feel
	comfortable and what I need to do to accept individuality without enforcing
	conformity.
\end{quote}
% On a related note, I would love to see the raw data from that survey.  It may
% help inform how I can help my own division in this area.
Many of our faculty commented on the use of group work as a way to expose
students to diverse ideas and culture.  They also indicated that they tried to
be culturally aware when writing application problems by choosing different
names, genders and roles for the characters in their problems. The `Rule of
Four' (functions and relations should be presented numerically, graphically,
verbally and symbolically) is incorporated into most of our CCOGs. The rule of
four recognizes and highlights the different ways people prefer to learn
mathematics.
\subsection[Educational cost]{Educational cost (addresses socio-economic condition)}
Our SAC is aware of the cost of course materials and considers the
socio-economic condition of our student population when selecting texts.

The SAC has a long-standing policy to require the same textbook for all
sections of a course. Since PCC has such a large student body and we offer many
math classes, this policy has enabled the math SAC to negotiate wholesale
prices of textbooks (particularly custom editions) with publishers.  This saves
students money when taking a sequence course (for example MTH 60/65) and allows
students to sell back their book to the bookstore.

Publishers can create a custom edition from an existing textbook by removing
material (e.g., chapters) or adding material (e.g., supplemental materials);
the publisher labels the textbook `A Custom Edition for Portland Community
College', and thus restricts its resale value, as it can only be used at PCC;
this benefits the publisher and enables them to reduce the price to PCC.

Math SAC subcommittees have successfully implemented this idea with the
textbooks for almost all of the mathematics classes taught at PCC: MTH 20, the
MTH 60/61/62/63/65 sequences, MTH 95, MTH 111, MTH 112, MTH 105, the MTH 243/244
sequence, and the MTH 251/252/253/254 sequence.

In addition to using custom editions uniformly across the district, we have a
group that is investigating an in-house Pre-Calculus text to reduce dependency
on publishers. The group is inactive at this point because they have been
unable to secure adequate release time; for more information, see
\cpageref{cur:sec:111/112doc} and \cite{mth111project}.

We actively pursue free and open source products such as WeBWorK---the fully
accessible online homework system (see \cpageref{other:sec:webwork}). This
meets our goal of providing low cost curricular materials and also supports
student access. The University of Oregon has generously hosted several WeBWorK
courses for PCC over the past few years.  Disability Services has provided
strong support for WeBWorK, and we were able to procure our own WeBWorK server
at PCC in the Fall of 2013.

\subsection{Educational background}
We have several projects and classes in place to address our students'
different educational backgrounds (\cpageref{cur:sec:other} details information on
some of our initiatives).

The Study Skills program (first discussed on \cpageref{cur:sub:studyskills}) was created to address the different educational
backgrounds of our students, particularly those students who have
underdeveloped study skills. This program consists of seven topics all relating
to study skills specific to mathematics: how learning math is different,
resources available for help at PCC, time management, listening and note-taking
skills, why and how to do homework, test taking strategies, and how to overcome
math and test anxiety.  Each lesson is broken up into three parts: a short
video to be watched by students outside of class, a student worksheet to be
completed in conjunction with the video, and an in-class discussion lead by the
instructor.

MTH 07/08 (also known as AMP) described on \cpageref{other:sec:amp} addresses differences in educational backgrounds.
It allows students who have previous exposure to the material to attempt to
move to a higher level class (see \vref{app:sec:ampdata}).

Even with the above mentioned programs, we feel that the college could do a lot
more when it comes to placing students into classes appropriately and orienting
them to the demands of college. We have formed the Placement Test Reform Group
described on \cpageref{other:sec:placementreform} and would like to see more wrap-around services
for students in developmental classes; these services would ideally begin
before the students steps into the classroom. We suggest the adoption of a
placement test that measures study skills, motivation and academic
preparedness.

\recommendation{We recommend that students who are not academically prepared be required to
	take a study skills course. We would like to see more math-specific advisors
	and have enough advisors so that it is feasible for a student to see an advisor
	every term.  We would also like to see the tutoring center open during the
	first week of the term. In our experience, students who are behind during the
first week have a hard time catching up.}
% Maybe mention the proposed CG course here?

\subsection{Data trends}\label{needs:sec:trends}
Despite the above mentioned efforts to have instruction informed by our student
demographics, we have still found that there is an achievement gap when it
comes to minority and underrepresented populations. We have displayed data for
five years in \vref{app:sec:demographicdata}; see
\crefrange{app:tab:demographic2008-2009}{app:tab:demographic2012-2013}. PCC has
undergone vast enrollment changes over the last five years since our previous
program review; here are the trends that we observed for this time period:
\begin{itemize}
	\item The percentage of both White and Asian students increases as students
	progress through the sequence of MTH classes. There appears to be a
	modest increase in diversity levels in MTH 251--254 over the last 5
	years, but this may be due to more students identifying as Multiracial.
	\item There is a slight increase in diversity since AY 2008 (the percentage
	of students who identify as white has decreased in most of our courses).
	\item There is a shockingly high number of students aged 19 or less who
	place into MTH 20. Since many of these students should have been exposed
	to the material from MTH 20 recently, we need to further examine both the placement
	exam and our communication with high schools. For example, are high
	schools allowing students to use calculators too freely? Are students who
	are otherwise proficient at algebra placing low due to not understanding
	fractions? This is something that needs further investigation. If we
	could decrease the number of young students placing into MTH 20, we might
	be able to shorten their path to a degree.
	\item The percentage of students aged 50+ decreases through the DE
	sequence. We suggest intentionally creating support systems for students
	aged 50+, particularly in MTH 60. These students likely have been out of
	the educational system the longest, so they face different challenges
	than their younger peers.
	\item There is a large decrease in the percentage of black students from
	MTH 20 to MTH 60. Not only is there a percentage decrease, there is also
	a decrease in the total number of black students in MTH 60 compared to
	MTH 20. This indicates that MTH 20 is likely a significant barrier to
	some minority students or that minority students place into MTH 20 at a
	disproportionately high rate. Although this is relatively consistent with
	national data, we would like the administration to continue to support
	programs like Passages, Project Independence, ROOTS, and other interventions
	to increase success rates of
	minority students. In addition, a more diverse faculty might help with
	retention and passing rates; \emph{ the extent to which [the instructor's attributes]
		differ from the physical, cultural, and intellectual backgrounds of [his/her] students
		will have a profound effect on the interactions in [the] classroom.}
	\footnote{\url{http://www.crlt.umich.edu/gsis/p3_2}}
	\item The pass rates for black students are noticeably lower each year and
	in each course.

	\recommendation{We recommend intentionally creating support systems for
	black students studying mathematics.}
	\item Females consistently pass MTH 20/60/65/95 at higher rates than males,
	but a smaller proportion of females enter MTH 112 and the calculus
	sequence, an important gateway to engineering careers.

	\recommendation{We
		suggest identifying ways to encourage female students to  continue on to
	MTH 112 and related STEM careers.}

	Other options that support students in
	mathematics may not come from the Math SAC itself. For example, the Math
	Club at Cascade serves as a good opportunity for females who enjoy
	mathematics to connect with and support eachother. However, such clubs
	are largely student led and participation can vary greatly from year to
	year and term to term.

	\item While female students are underrepresented in MTH 112, and MTH 251--254
	as noted above, it does appear that many female students take a statistics
	route (MTH 243, 244) instead of a calculus route (MTH 251 and above). MTH 243
	and 244 lead toward many important career paths and is more direcly relevant
	to students interested in these paths. Many female students are by-passing
	the calculus route and entering 200 level mathematics directly instead. The
	Math SAC recently voted to change the prerequisite for MTH 243, which could
	open and shorten this pathway for many more students (male and female alike)
	in years to come.

	\item The percentage of men passing MTH 20 is lower than that of female
	students. In addition, it appears that the percentage of males enrolling
	in MTH 20 is increasing (perhaps due to the economic downturn). This is
	consistent with data at the secondary level. Since MTH 20 is pre-algebra,
	some of this may be due to prior educational experiences and students
	attitudes of their ability.
\end{itemize}

\section[Changes in instruction due to changes in demographics]{Have there been any notable changes in instruction due to changes in
demographics since the last review?}
At Cascade, the number of MWF classes has increased since the last program
review. This was done in response to the increased demand for MTH 61/62/63. The
increase in these classes seemed to coincide with a large influx of
under prepared students who returned to school after the recession.

Classes that run three days a week are designed to help students who struggle
with the demands of a two-day-a-week class.  While there isn't a notable
difference in success rates between MWF classes and those that meet less
frequently, it is felt that the shorter class time is better for students
cognitively, their attention span is held longer and students engage in more
frequent practice of mathematics.

We would like to see more MWF or even MTWTh classes to provide more flexible
scheduling options for the benefit of students. We suggest that one way to
accomplish this is to turn more MW classes into MWF classes.

\section[Demand and enrollment patterns]{Describe current and projected demand and enrollment patterns.
 Include discussion of any impact this will have on the program/discipline.
}
Demand and enrollment patterns have been divided into two categories:
Developmental and Lower Division Transfer Mathematics.

\subsection{Developmental mathematics}
Referring to \cref{needs:fig:enrollmentDevelopTerm}, we see that
enrollment in pre-college courses increased from AY2008 to AY2011 by 46\%;
there was a slight decrease (5\%) in enrollment from AY2011 to AY2012.

\begin{figure}[!htb]
	\centering
	% arara: pdflatex
% !arara: indent: {overwrite: yes}
\documentclass{standalone}
% Caption: Enrollment in Developmental MTH by TERM
% 2008-2012

\usepackage{pgfplots,pgfplotstable}
\pgfplotsset{compat=newest}

\begin{document}

\pgfplotstableread{
	Year    Summer  Fall    Winter  Spring
	2008  1964    6322    6085    5962
	2009  2586    7657    8008    7565
	2010  3330    8289    8104    7683
	2011  3284    9148    8944    8269
	2012  3097    8946    8467    7531
	}\mydata

\begin{tikzpicture}
	\begin{axis}[
			%ybar,
			symbolic x coords={2008, 2009, 2010, 2011, 2012},
			xtick=data,
			minor ytick={1000,2000,...,10000},
			xticklabels = {2008/09,2009/10, 2010/11, 2011/12, 2012/13},
			enlarge x limits,
			scale only axis,       
			xticklabels = {2008/09,2009/10, 2010/11, 2011/12, 2012/13},
			grid = both,
			ymin=0,ymax=10000,
			scaled ticks=false, 
			tick label style={/pgf/number format/fixed},
			legend pos= outer north east,
			width=0.6\textwidth,
			x tick label style={rotate=25},
		]
		\addplot table[x=Year,y=Fall]{\mydata};
		\addplot table[x=Year,y=Winter]{\mydata};
		\addplot table[x=Year,y=Spring]{\mydata};
		\addplot table[x=Year,y=Summer]{\mydata};
		%\addplot table[x=Year,y expr=\thisrow{Summer}+\thisrow{Fall}+\thisrow{Winter}+\thisrow{Spring}]{\mydata};
		\legend{Fall, Winter, Spring,Summer, Total}
	\end{axis}
\end{tikzpicture}
\end{document}

	\caption{Enrollment in Developmental MTH by Term}
	\label{needs:fig:enrollmentDevelopTerm}
\end{figure}

The counterpart
to \cref{needs:fig:enrollmentDevelopTerm} \emph{by campus} is given in \cref{needs:tab:enrollmentDevelp}
(see also \cref{app:fig:enrollmentDevelopCampus} in \vref{sec:app:enrollment}).
Enrollment in Developmental mathematics courses increased most
at CA and SEC; from AY 2011 to AY 2012, enrollment in Developmental mathematics courses
decreased at all campuses except SEC. In many cases,
sections that we would have liked to offer encountered scheduling difficulties in 2011/12 due to a lack 
in facility space---this is discussed further in  \vref{facilities:sec:scheduling}.

\begin{table}[!htb]
	\begin{widepage}
	\centering
	\caption{Developmental Mathematics Enrollment by Campus}
	\label{needs:tab:enrollmentDevelp}
	\begin{tabularx}{\linewidth}{X*{7}rr}
		\toprule
		    &         &         &         &         &         & \% change  & \% change  & \%change  \\
		    &         &         &         &         &         & 2008/09-- & 2011/12--  & 2008/09-- \\
		    & 2008/09 & 2009/10 & 2010/11 & 2011/12 & 2012/13 & --2011/12  & --2012/13  & --2012/13 \\
		\midrule
		SY  & 6764    & 8155    & 8847    & 9682    & 8840    & 43.14\%    & $-8.70\%$  & 30.70\%   \\
		CA  & 4159    & 5745    & 5963    & 6585    & 5887    & 58.33\%    & $-10.60\%$ & 41.55\%   \\
		RC  & 6625    & 8033    & 8192    & 8669    & 8454    & $30.85\%$  & $-2.48\%$  & 27.60\%   \\
		ELC & 2785    & 3883    & 4404    & 4709    & 4860    & $69.08\%$  & 3.21\%     & 74.50\%   \\
		\bottomrule
	\end{tabularx}
	\end{widepage}
\end{table}
% is this really what you want to highlight?  Personally I am curious what the
% overall change is from 2008 to 2012 (including that last year of downturn).
% You could include that in addition to the percent downturn between 2011 and
% 2012.  Can you possibly add this new column I suggest in?  You have the data
% you need right there to calculate it.
%CV Fixed. Double check that it looks right to you though.
% yep, looks good. cmh 1/18/14

\subsection{Lower division transfer mathematics}
Enrollment in Lower Division Collegiate (LDC) courses increased over the 5
year period of this Program Review, but at a decreasing rate (i.e. the graph is concave down). This means that the overall number of students increased each year, but the rate of increase each year was smaller and smaller. The increase in LDC enrollment could possibly be due to students looking toward PCC as a less expensive alternative to our 4 year counterparts. It could also be a ``bubble'' of students who came to us during the economic downturn working their way up through our math sequence.  There was a 36\% enrollment increase from Summer 2011 to Summer 2012. We suspect this is due to
changes in financial aid eligibility. Prior to this change, students were
awarded financial aid for fall, winter, and spring, and needed a separate
application for summer term.  After the change in eligibility, students were
awarded aid for an entire academic year, commencing with Summer 2012.
This increase in enrollment is shown in \cref{needs:fig:enrollmentLDCTerm} with
its per-campus counterpart in \vref{app:fig:enrollmentLDCCampus}.


\begin{figure}[!htb]
	\centering
	% arara: pdflatex
% !arara: indent: {overwrite: yes}
\documentclass{standalone}
% Caption: Lower Division College Transfer enrollment by term
% 2008-2012

\usepackage{pgfplots,pgfplotstable}
\pgfplotsset{compat=newest}

\begin{document}

\pgfplotstableread{
Year	Summer	Fall	Winter	Spring
 2008	1220	2425	2532	2627
 2009	1484	3068	3226	3250
 2010	2037	3450	3538	3494
 2011	1911	5446	5238	5202
 2012	2599	5513	5242	5189
	}\mydata

\begin{tikzpicture}
	\begin{axis}[
			%ybar,
			symbolic x coords={2008, 2009, 2010, 2011, 2012},
			xtick=data,
			minor ytick={1000,2000,...,10000},
			enlarge x limits,
			scale only axis,       
			xticklabels = {2008/09,2009/10, 2010/11, 2011/12, 2012/13},
			grid = both,
			ymin=0,ymax=10000,
			scaled ticks=false, 
			tick label style={/pgf/number format/fixed},
			legend pos= outer north east,
			width=0.6\textwidth,
			x tick label style={rotate=25},
		]
		\addplot table[x=Year,y=Fall]{\mydata};
		\addplot table[x=Year,y=Winter]{\mydata};
		\addplot table[x=Year,y=Spring]{\mydata};
		\addplot table[x=Year,y=Summer]{\mydata};
		%\addplot table[x=Year,y expr=\thisrow{Summer}+\thisrow{Fall}+\thisrow{Winter}+\thisrow{Spring}]{\mydata};
		\legend{Fall, Winter, Spring, Summer, Total}
	\end{axis}
\end{tikzpicture}
\end{document}

	\caption{Enrollment in LDC, College Wide, by term}
	\label{needs:fig:enrollmentLDCTerm}
\end{figure}

In particular, five-year enrollment increases  in LDC are large at all
campuses, as shown in \cref{needs:tab:LDCenrollmentCampus}.  We expect the
increase would be larger at SY if not for lack of facilities space. A lot of
this growth is in the Calculus sequence.

\begin{table}[!htb]
	\centering
	\caption{LDC enrollment by campus}
	\label{needs:tab:LDCenrollmentCampus}
	\begin{tabular}{l*{5}{c}r}
		\toprule
		    &        &        &        &        &        & \% Increase \\
		    & 2008/09 & 2009/10 & 2010/11 & 2011/12 & 2012/13 & 2008/09--2012/13   \\
		\midrule
		SY  & 4096   & 4883   & 5405   & 7173   & 7297   & 78.15\%     \\
		CA  & 1497   & 2036   & 2042   & 3155   & 3435   & 129.46\%    \\
		RC  & 2920   & 3625   & 4451   & 6262   & 6424   & 120.00\%    \\
		ELC & 291    & 484    & 621    & 1207   & 1387   & 376.63\%    \\
		\bottomrule
	\end{tabular}
\end{table}

\subsection{Totals (DE and LDC combined)}
Overall enrollment increased from AY2008 to AY2011 by 63\%. This significant
increase reflects the downturn of the economy five years ago. Many students
returned to school because their jobs had ceased to exist or they hoped to
better their chances of employment with a degree or certificate. There was a
slight decrease (2\%) in overall enrollment from AY2011 to AY 2012, which is
mainly due to a decrease in enrollment in developmental mathematics courses
(see \vref{app:fig:totalenrollmentTerm}).


Each campus experienced slightly different enrollment trends. Enrollment
increases at CA and SEC from 2008--2011 were significantly higher than other
campuses. SEC experienced a 92.33\% increase in enrollment over the 4 year
period and a continued enrollment increase from AY2011 to AY2012. RC
experienced the lowest \% drop (of campuses whose enrollment dropped)  in
enrollment from AY2011 to AY2012 (see \vref{needs:tab:enrollmentcampusyear} and
\vref{app:fig:totalenrollmentCampus}).

\begin{table}[!htb]
	\centering
	\caption{Enrollment by campus and year}
	\label{needs:tab:enrollmentcampusyear}
	\begin{tabular}{l*{5}{c}rr}
		\toprule
		    &        &        &        &        &        & \% change & \% change \\
		    & 2008/09 & 2009/10 & 2010/11 & 2011/12 & 2012/13 & 2008/09--2011/12 & 2011/12--2012/13 \\
		\midrule
		SY  & 10860  & 13038  & 14252  & 16855  & 16137  & 55.20\%   & -4.26\%   \\
		CA  & 5656   & 7781   & 8005   & 9740   & 9322   & 72.21\%   & -4.29\%   \\
		RC  & 9545   & 11658  & 12643  & 14931  & 14878  & 56.43\%   & -0.35\%   \\
		ELC & 3076   & 4367   & 5025   & 5916   & 6247   & 92.33\%   & 5.59\%    \\
		\bottomrule
	\end{tabular}
\end{table}

Furthermore, while enrollment (number of students) has increased over a
five-year period by 60\%, the number of sections offered has not kept pace
(only increased by 40\%) as detailed in \cref{needs:tab:averageclasssize}; we
are concerned that the average class size is increasing.

\begin{table}[!htb]
	\centering
	\caption{Average class sizes (district wide)}
	\label{needs:tab:averageclasssize}
	\begin{tabular}{lS[table-format=2.2]}
		\toprule
		Year   & {Average Class Sizes } \\
		\midrule
		AY2008 & 24.87                  \\
		AY2009 & 27.64                  \\
		AY2010 & 27.36                  \\
		AY2011 & 28.7                   \\
		AY2012 & 28.4                   \\
		\bottomrule
	\end{tabular}
\end{table}

While the average class size is somewhat small, there are classes that are much
larger than the average. There is little consistency between campuses when it
comes to class size, which seems to be determined almost entirely by room
choice; previous attempts at setting SAC-wide class sizes were not accepted by
the Deans of Instruction. We have resubmitted our report and are awaiting a
response---the report \label{needs:page:classsize} can be viewed in full in \vref{app:sec:classsize}.

The ratio of Pre-College to LDC enrollment has decreased (see
\vref{app:fig:ratioDevelopToLDC}). We are unsure of the reason for the decline
in this ratio, but we are concerned that our completion rates have decreased,
especially in developmental mathematics. Our overall success rates have
decreased with increased enrollment.  This could partially be explained by the
large number of under prepared students who entered the institution as a result
of the economic recession---see \vref{app:tab:successratesbyterm}.

Of concern is that while enrollment has increased from 2008 to 2013, hiring of
full-time faculty has not kept pace. In addition, the demands on full-time
faculty (subcommittee work, LAS, CIC,  etc.) have increased.   However,
if the economy continues to improve, it is expected that enrollment will level
off or slightly decline for a short period of time.

Since Governor Kitzhaber's proposal for the state of Oregon to have 40\% of
adults earn a bachelors or higher degree, 40\% with an Associates, and 20\%
with a high school or equivalent degree there is concern about how we will handle
enrollment demands as policies to meet these goals are implemented. In addition, the state mandate that students graduate with 9 college credits on their transcript means additional demands on faculty to coordinate dual credit programs, middle college and other programs designed to give high school students college credit. Classroom space, faculty workload, class size and student preparedness are all
major concerns.


\section[Strategies to facilitate access and diversity]{What strategies are used within the program/discipline to facilitate access and diversity?}\label{needs:sec:access}
The MTH SAC uses several strategies to facilitate access; for
example, we offer all-day classes, hybrid classes, evening classes, distance learning classes
and some weekend classes. This allows students who aren't available for traditional
weekday classes to access the mathematics program at PCC. We offer many classes (MTH 60, 65, 95, 111, and 243) in either a weekend hybrid format or all day Saturday class.


We facilitate access to students who learn differently or would like a
different learning structure by offering  Alternative Learning Center (ALC)
self-paced math classes (see \vref{app:sec:alc}).

We offer MTH 07/08  to returning students who are not happy with their
placement exam scores. This one-week intensive math review program is designed
to help students place into higher level MTH courses which saves students time
and money. It also facilitates quicker access to a degree if students are able
to place higher and shorten their time in the developmental sequence.  Even if
students do not place into a higher course, it seems to help students fill in
gaps in their knowledge.

In Fall 2012 the MTH SAC completed a large project centered around accessibility of online content
for students with disabilities. During Fall 2011, Mathematics faculty members
had realized that our subject matter presented unique complications not faced by other
disciplines. Chris Hughes and Scot Leavit were granted release time to
investigate accessibility as it applies to mathematics. We are grateful for the
support and the collaborative nature of the project.\label{needs:page:disabilityservices}
The full text of the report, including recommendations made to the SAC, can be found at
\cite{accessibilityproject} and a summary is given in
\vref{app:sec:accessibility}.

As a result of the project, faculty awareness of accessibility has increased
significantly. It has also lead to discussions surrounding adoption of
commercial online homework management systems (see details of the ALEKS pilot
in \vref{app:sec:aleks}) and an increased acceptance of WeBWorK, which is currently
the only fully accessible online homework
management system. There has been considerable work done to develop problem
libraries for Math 60/65 in WeBWorK that match our curriculum---the details are
discussed in \cpageref{other:sec:webwork}.

Most faculty feel that we need further education on how to facilitate access
and diversity in mathematics classes. In particular we would like further
discussion and training on the various types of accessibility challenges that
we face, including physical accessibility, learning accessibility, cultural and
social accessibility, and access to education.
\fixthis{comment below}
% page 46, paragraph 4:  It feels like this paragraph belongs at page 40, right
% before the Educational Costs section (that is to say ending and summing up
% the section on Individual Faculty Awareness.
%CV Chris, can you fix this? I don't know what the Educational Costs Section is.
%SS  Wherever it ends up, it needs to be followed up by a recommendation.
% cmh: phone call needed

\subsection{Physical accessibility}
Many of our instructors have some experience serving students that have either
a visual or hearing disability. We appreciate the continued relationship and
communication with the disabilities services on this issue. Most of our issues
of physical accessibility within the classroom are handled well by disability
services.

However, it is worth noting that on a SAC level, we face issues of
accessibility in the facilities we use for our meeting space. As one of the
largest SACs in the district, we often have trouble finding room space that is
large enough to accommodate our group and is accessible at the front of the
room. It is our understanding that the bond renovations will mostly address
this issue.

\subsection{Learning accessibility}
We would like continued training on how to provide equally effective
instruction for students with learning disabilities. We think that quarterly
workshops (perhaps offered through disability services) could help us to be
proactive and learn about methods that are specific to the teaching of
mathematics.

\subsection{Cultural and social accessibility}
Given our observations on \cpageref{needs:sec:trends}, we realize that we need
further education, research, and training in strategies to provide equally
effective instruction to students of all cultures, genders, and other facets
(see our definition of diversity on \cpageref{needs:sec:definitiondiversity}).

\subsection{Access to education for historically underrepresented populations}
We need more support to facilitate topics of social justice in a mathematics classroom. We
feel that students should know how to use mathematics as a tool for social
change; forming our Social Justice Workgroup (\cpageref{cur:sub:socialJustic}) is
a step in the right direction. We are currently working on disseminating the activities and the discussions from this
group and sharing them with the larger SAC.

\section[Approved academic accommodations]{Describe the methods used to ensure faculty are working with Disability Services to implement approved academic accommodations?}
During the 2012/2013 academic year, the Office of Students with Disabilities
went to a paperless notification system for all academic accommodation
notifications.  Initially, this change to email notifications led to several
problems.  One problem arose from the fact that not all faculty, especially
part-time faculty, are as diligent in monitoring their PCC email as is
necessary to make an email notification system work efficiently.  More
problematically, there were glitches in PCC's email system that led to many of the
notifications being misdirected to quarantine or spam folders.  As a result,
there were students who had approved accommodations of which their instructors
were unaware.

To help remedy this situation, the mathematics department chairs contacted the
Office of Students with Disabilities and asked if there were some way that a
back-up system could be created to catch notifications that have fallen through
the cracks.  In response, Kaela Parks (director of the OSD) has created a
spaces page that lists every course in which there is at least one student
enrolled who has approved accommodations.  Kaela had the foresight to create a
page that updates in real time.  For example, if a student has made an
accommodated testing request but the instructor has not yet completed the
accommodated testing form, that class is flagged in red and the relevant
department chair can contact the faculty member to let them know about the
situation.

There had been growing concern among several faculty members
about the nature of many accommodations including:  calculator usage that
contradicts assessment criteria stated in CCOGs, and  those that require flexibility
in due dates (which can lead to the withholding of keys for other students).  Kaela Parks came to a mathematics SAC
meeting to discuss these concerns, at which time she reiterated the concept of
reasonable accommodations---that accommodations for student disabilities are not
meant to compromise student learning outcomes.
Kaela also said that
faculty can always contact her or the student's assigned OSD counselor to
discuss specific accommodations of concern.

\section[Curriculum/instructional changes due to internal/external feedback]{Has feedback from students, community groups, transfer institutions,
 business, industry or government been used to make curriculum or instructional
 changes (if this has not been addressed elsewhere in this document)?  If so,
describe. }
Mathematics support for Career Technical Education (CTE) has evolved over the
years.  Currently CTE students take mainstream math courses to fulfill their
math requirement;  this concerns PCC, other academic institutions, and
officials in the State of Oregon as it is not obvious what benefit can be gained
by taking courses meant for students destined to take Calculus classes. Current issues involve: should PCC create
math courses focused on CTE students only, how would students  transfer between
these courses and the general mathematics curriculum.   The following is taken
from \cite{natcentereduc}.
\begin{quote}
	The research we did revealed a major gap in the alignment between the
	mathematics courses taught in the mathematics departments in our community
	colleges and the mathematics actually needed to be successful in the applied
	programs students are taking.
\end{quote}
Research to develop a CTE-MTH alternative track was underway but has stopped
and be repurposed to fit our changing vision of DE math pathways.  The hope is that a STEM-focused  pathway will provide more meaningful content for all students, regardless of their goals.
If we are not successful in our attempt to create the pathways developed during the NSF-IUSE initiative, we believe that creation of an independent pathway targeted specifically to CTE-MTH students should be pursued.  We believe that such a pathway would increase completion rates for
CTE students. A CTE-MTH pathway would also address concerns from CTE programs that our classes
do not properly prepare their students for the mathematics that is used in
their programs.  

Finally, the MTH 243 curriculum credit hour change detailed on \cpageref{other:sec:mth243}
was initiated by student feedback.
