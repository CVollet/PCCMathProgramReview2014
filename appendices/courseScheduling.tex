% arara: pdflatex: {files: [MathSACpr2014]}
% !arara: indent: {overwrite: yes}

\chapter{Course Scheduling Pattern (by campus)}\label{sec:app:courseschedule}
\section*{Cascade}
\begin{enumerate}
    \item  Scheduling is term by term, which helps us adjust to enrollment
      changes and part-time faculty changes.
    \item  Class size for all Cascade math classes is capped at 35 (if room
      allows) except MTH 20/61/62/63, which are capped at 30.
    \item  Since the last program review, we have regularly offered many more
      MWF classes, especially for MTH 95, in order to try to improve retention
      and success.  MWF classes, meeting for shorter times than typical MW/TuTh
      classes, enable us to `pack' more classes into a school day and therefore
      maximize our usage of the rooms we are assigned.
    \item  We discontinued MTH 91/92 because we felt that the sequence was
      inadequately preparing students for MTH 111.
    \item We discontinued offering MTH 20 DL because student success rates were 
      lower than in the face-to-face format.
    \item  We continued to innovate with regard to hybrid offerings, including
      weekday and Saturday hybrids.
    \item  We eliminated Sunday hybrids when Cascade decided to eliminate
      Sunday class offerings.  Since we were beginning to see declines in
      enrollment anyway, this did not seriously impact student access to
      classes.  The Saturday hybrids are still available.
\end{enumerate}

\section*{Rock Creek}
\begin{enumerate}
    \item Rock Creek schedules term by term.  It would help with staffing
      decisions if the classes would be assigned rooms well ahead of the date
      the class offerings become visible to students online and the deadline
      for the photograph proof of the paper class schedule.
    \item Rock Creek offers mostly two day a week classes (82\%) meeting from
      7am to 9  pm, about 10\% one day a week either Saturday or Friday
      mornings, and 8\% online.
    \item Rock Creek schedules courses at the Hillsboro Center, Willow Creek
      Center and St Helens  (12\% of class offerings at RC).
\end{enumerate}

\section*{Southeast}
\begin{enumerate}
  \item Southeast assigns classes term by term, due to having mostly part-time
    faculty who need the flexibility of when to teach and at what times of day.
  \item Class size caps for math classes at SE is usually 30 or 35, depending
    on the size of the room and the room availability.  
  \item Most classes are offered on a Mon/Wed or Tue/Thu schedule, with a good
    balance of morning, afternoon, and evening classes.
  \item Southeast offers the full range of mathematics courses; however, most
    of the students at SE take 20, 60, 65, or 95.  Slowly but surely, there
    have been more students taking 111 and 112, along with calculus and beyond.
  \item There is a substantial offering of DL classes, mainly in 60, 65, 95,
    and 243.  Over the past year, 111 was added to the mix, while 20 was
    removed from the offerings (due to abominable passing rates).
\end{enumerate}

\section*{Sylvania}
\begin{enumerate}
    \item  Scheduling is done one year ahead, which helps students plan out
      their year.
    \item  Coordination between campuses for low enrollment or specialty
      courses.
    \item  Newberg Center, scheduled by Sylvania, gives more students better
      access.
    \item  Increased offering of Distance Learning courses also increases
      accessibility for students with scheduling conflicts.
	\item  Class size for all Sylvania math classes is capped at 34 (if room allows) except
    Statistics (23-28 for computer classrooms).
	\item  Reorganized the time slots for 2013/14 which should lower possibility of canceled classes (due 
	to room availability or low enrollment).
\end{enumerate}

