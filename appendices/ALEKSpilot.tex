% arara: pdflatex: {files: [MathSACpr2014]}
% !arara: indent: {overwrite: yes}
\chapter{ALEKS pilot}\label{app:sec:aleks}
\section{MTH 20 Several classes during 2012--2013 AY (Edwards)}

The pilot includes the extensive use of ALEKS, a technology based assessment learning system,  in 2 on campus and 2 online classes each term.  

Course logistics: 
\begin{itemize}
    \item Students are walked through an introduction to the system and given
      an assessment.
    \item Students are then provided with a very clear visual pie chart showing
      them what they know.
    \item ALEKS then provides students the opportunity to work on a range of
      instructor chosen topics at their current level.  Student only work on concepts they have not mastered.
    \item Explanations and videos are provided with each topic.
    \item Students are provided instant feedback and instant online teaching.
    \item Students are not given the option to skip work that they have not
      mastered, essentially forcing them to learn the material and 
	fill in the concepts gaps that they began the class with.
    \item Students are routinely assessed with new topics available as they
      move through the course.
    \item Students are in the computer lab working on ALEKS throughout the
      class period.
    \item Students (generally for whom the material is recent) have the
      ability to move ahead.  
\end{itemize}

\subsection{Results and Statistics}
I'm reflecting only Fall term math 20 students; this was a definite pilot.  
A variety of changes were incorporated into Winter and Spring terms 
which included additional lectures and assignments that had each class 
more closely resemble more traditional class.
\begin{itemize}
	\item Students loved the instant feedback.
    \item Students enjoyed the ability to work in the ALEKS system, choosing
      their topics, and getting ahead when desired.  There were very, very few
      complaints about the system.
    \item Students became aware of how much time they studied, with a clear
      visual of the relationship between study time and learning. 
    \item FOUR students last term completed the math 20 material, moved on to
      math 60 material, took and passed my math 60 final exam.
\end{itemize}

On Campus Classes:  
\begin{itemize}
    \item 78\% of students passed math 20 last Fall compared to 89\% using
      ALEKS  (7am class result was 63\% passed using ALEKS).
    \item Of those that went on to math 60:  60\% passed last Fall compared to
      69\% using ALEKS (7am class result: 13\% passed, 1 in 8).
\end{itemize}
DL Classes:
\begin{itemize}
    \item 62\% of students passed math 20 last Fall compared to 71\% using
      ALEKS.
    \item Of those that went on to math 60:  61\% passed last Fall compared to
      46\% using ALEKS.  
\end{itemize}

\section{Pilot in Math 112 during Winter 2013 (Louie)}
\fixthis{this paragraph is very long}
I think the most beneficial part about ALEKS is the instant feedback and
instant teaching. This gives the student a chance to fill in the holes of their
knowledge. My data was from a very small group. I compared one class (no aleks)
to 2 classes (with classes). I averaged data from the 2 classes to get more
accurate results. Surprisingly the grade distribution and overall pass rate was
very close from ALEKS to no ALEKS. The distribution of grades was also very
similar. In both classes my pass rate was 73\% which is well above the current
57\% campus average pass rate. 

The attrition rate for non ALEKS classes was 32\%.  The ALEKS class averaged a
mere 14.7\%. Does ALEKS keep students on task and less likely to
withdrawl from the course? The numbers seems to support it but the sample size
was small. The other benefit to ALEKS is requiring students to do homework and
keeping track of their progress. The max average time spent on ALEKS was 15.4
hours and the minimum 1.6 hours per week. Despite my lack of data to support
higher grades, I am confident that the students should be more prepared for
Math 251. I am planning to check the success rates of the students who went on
to calculus at Cascade campus. I would like to see if the pass rate of ALEKS
students is higher than that of NON-ALEKS courses. Data is still in the works.

Best of all, students were forced to complete ALL homework and lectures seemed to flow with little interruption. I have not completely compiled the results from an ALEKS survey I gave at the end of the term. However the beginning results favor that most students enjoyed using ALEKS for homework and found it helpful in their learning.
