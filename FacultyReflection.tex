% arara: pdflatex: {files: [MathSACpr2014]}
% !arara: indent: {overwrite: yes}
\chapter[Faculty composition and qualifications]{Faculty:  reflect on the
composition, qualifications and development of the faculty}
\epigraph{ %
I want to tell you what went through my head as I saw the D on my 3rd exam in
Calc II:  ``f\#\$king a\&\$hole!''  That D changed my life.  The feeling of
failure, not from my incompetence but rather my laziness.
I want to let you know that every single success in my life now is due in part
to your teachings.  I can't thank you enough \& I hope that if for nothing
else, you have made a great influence on me.}{Oke Tammik, PCC Mathematics
Student, December 2013}

%\epigraph{Professional development is particularly important for developmental
% education instructors, as these individuals tend to have limited previous
% training for teaching basic skills students. Unfortunately, studies have
% found that most community 
% colleges provide only episodic staff development activities, which tend to
% take the form of one-day workshops or seminars led by outside experts, or
% else isolated conversations among colleagues or departmental meetings that
% focus on logistics or content knowledge rather than pedagogy. Sadly, studies
% have revealed that such isolated professional development does little to
% change individuals' everyday practice, as they become ubsumed in normal
% routines and have little support for integrating new learning into their
% practice.}{(Rutschow, E.Z. \& Schneider, E. 2011. Unlocking the gate: What we
% know about improving developmental education. Oakland, CA: MDRC, p. 66)}

\section[Faculty composition]{Provide information on each of the following:}
\subsection[Quantity and quality of the faculty]{Quantity and quality of the faculty needed to meet the needs of the
program or discipline.}
The total number of full time faculty at all campuses between 2011 and 2013
varied between 36 to 41 and the 
part time faculty varied between 143 to 158 on any given term,  not including
Summer.     The percent of \emph{all} courses (pre-college and college level)
taught by full-time instructors during this time period varied from a low of
24.9\% at Cascade to a high of 41\% at Sylvania (see
\cref{reflect:tab:percentallcourses}).

From the academic year 2008/09 to 2012/13 there was a significant increase in the
number of students taking math courses at all campuses as shown
\cref{reflect:tab:enrollment}. 
\begin{table}[!htb]
  \begin{widepage}
	\begin{minipage}[t]{.4\textwidth}
		\centering
		\caption{Percentage of courses taught by full-time faculty from 2011--2013}
		\label{reflect:tab:percentallcourses}
		\begin{tabular}{rrrr}
			\toprule
			SY   & RC     & ELC  & CA     \\    
			41\% & 28.2\% & 26\% & 24.9\% \\
			\bottomrule
		\end{tabular}
	\end{minipage}%
	\begin{minipage}[t]{.6\textwidth}
		\centering
		\caption{Enrollment Difference from AY 08/09 to AY 12/13}
        \label{reflect:tab:enrollment}
		\begin{tabular}{lrr}
          \toprule
			Campus & Enrollment Difference & \% increase \\
            \midrule
			SY     & 5277                  & 48.59\%     \\
			CA     & 3666                  & 64.82\%     \\
			RC     & 5333                  & 55.87\%     \\
			ELC    & 3171                  & 103.09\%    \\
            \bottomrule
		\end{tabular}
	\end{minipage}
  \end{widepage}
\end{table}

\Cref{app:tab:analysisPTFT} summarizes the breakdown of courses taught by
full-time and part-time faculty from Summer 2011--Spring 2013; breakdowns 
by \emph{term} are given in \vref{app:sec:analysisPTFT}.

\begin{table}[!hb]
	\centering
	\caption{Summary of sections taught (by campus) from Summer 2011--Spring 2013}
	\label{app:tab:analysisPTFT}
	% summary
	% summary
	% summary
	\pgfplotstableread[col sep=comma]{./data/sectionsTaughtPTFT/sectionsTaughtPTFT2011-2013.csv}\sectionsTaughtSummary
	\pgfplotstabletypeset[sectionFTPT]{\sectionsTaughtSummary}
\end{table}

In reference to ``quality of the faculty needed to meet the needs of the discipline,''  it is insufficient to look at degree or experience qualifications alone.  Even a short list of what we expect from our mathematics faculty would include, but not be limited to that she/he
\begin{itemize}
  \item possess an understanding of effective mathematics teaching methodologies and      strategies, and be able to adjust in response to student needs;
  \item  teach  the course content as outlined in CCOGs and with the appropriate mathematical  rigor;
\item show genuine commitment to students' success;
\item identify problems when students encounter difficulties learning;
\item demonstrate an ongoing intellectual curiosity about the relationship
  between teaching and learning;
\item manage classroom learning environments and effectively handle student
  discipline problems;
\item demonstrate technological literacy needed in the teaching of mathematics;
\item participate in professional organizations;
\item develop, evaluate and revise the mathematics curricula;
\item serve and contribute to the PCC community as a whole through campus and
  district wide committees and activities.
\end{itemize}
In addition, with the enormous enrollment increases of the past several years,
there are more students than ever needing both remediation in mathematics and
guidance in general about what it takes to be a successful college student.

Addressing this section heading directly, the `quantity' of full-time faculty
needed to achieve the `quality' goals noted above is currently inadequate.  It
is primarily the full-time faculty that has the time, resources and
institutional support to fully realize the expectations noted above.  Part-time
faculty are dedicated, but the expectations are different given the
many of the challenges they face (discussed below).   To increase the probability that a student moves successfully
through our mathematics courses without sacrificing quality, having a larger
full-time faculty presence than currently exists is needed.



In recognizing the need for more full-time faculty, we do not want to downplay
the skills and talents of our part-time faculty.  We have approximately 150
part-time instructors that serve our students each term, many of whom have
teaching experience from other colleges and universities; they bring additional
experiences from industry, other sciences, high school and middle school
education, and so much more.  Since they teach such a high percentage of our
classes, their success is crucial to our students' success.

Given the importance of part-time faculty, efforts needs to be made to minimize the many challenges that are unique to them.  Many of these challenges are created by the fact that part-timers frequently work on more than one campus or have a second (or third) job beyond their work for PCC.  Many of the problems are created by the institution itself.  The challenges include limited office space, limited access to office computers and other resources, limited opportunities to attend meetings, limited opportunities to engage in professional development activities, limited opportunities for peer-to-peer discourse.   

\recommendation{The college should create a task force to find ways to minimize the challenges faced by part-time faculty.  Given the heavy reliance on part-time faculty for staffing our courses, there is  little chance that we can institutionalize significant changes in our DE courses without an empowered part-time work force.}

\recommendation{The college should allow mathematics departments, at the discretion of each campus' faculty, to hire full-time faculty who meet the approved instructor qualifications for teaching at the pre-100 level but not the approved instructor qualifications for teaching at the post-100 level.  A large majority of our courses are at the DE level, and the needs of students enrolled in those courses are frequently different than the needs of students enrolled in undergraduate level courses.  Having a robust assortment of full-time faculty educational experience can only help in our pursuit of increased student success and completion.}


\subsection[Faculty turnover]{Extent of faculty turnover and changes anticipated in the next five
years.} 
Since 2011, ten full-time instructors have been hired and seven full-time
instructors have left campuses across the district (this includes full-time 
temporary positions).  Of the seven full-time
instructors who left, five retired, one left to pursue other job opportunities,
and one returned to another teaching job after her temporary full-time position terminated.  Three of the retirements occurred at Sylvania and one each at Rock
Creek and Cascade.  In addition to those that left the college, four full-time instructors transferred from cone campus to another.   Given no unexpected events, we anticipate that these demographics will roughly be repeated over the next five years.


Since 2011, 53 part-time instructors have been hired and 35 part-time
instructors have left campuses across the district.  Of the three campuses,
Rock Creek has the most part time faculty turnover, followed by Cascade and
Sylvania.  Reasons for leaving varied, but at least eight of the part-time
instructors who left campuses simply moved to another campus in the district
(see \vref{app:sec:facultyDegrees}).

\subsection[Part-time faculty]{Extent of the reliance upon part-time faculty and how they compare
with full-time faculty in terms of educational and experiential backgrounds.}
Across the district, the mathematics departments rely heavily upon part-time
faculty to teach the majority of the math classes offered.  Between 2011 and
2013, 75.1\% of the classes at Cascade were taught by part-time instructors,
71.8\% at Rock Creek, 72.7\% at Southeast, and 59\% at Sylvania.  This reliance
on part-time faculty to teach classes has been a challenge to the departments
in a number of ways:  
\begin{itemize}
\item the turnover of part-time faculty is higher and
thus there is a need to orient new employees more frequently and provide
mentoring and guidance to them as well;
\item many part-time faculty are on
campus only to teach their courses, and thus often do not attend meetings and
keep up with current SAC discussions on curriculum.  
\end{itemize}
For these reasons, classes have a higher probability to be taught with less consistency than the
mathematics SAC would like.  Increasing the number of full-time faculty (and
thus decreasing the dependence on part-time faculty) would mitigate much of
this inconsistency; complete details are given in  \vref{app:sec:analysisPTFT}.

Part-time faculty educational backgrounds vary much more than the full-time
faculty backgrounds.  Full-time instructors have master's or doctorate degrees
in mathematics or related fields with extensive math graduate credits.  About a
quarter of the part-time instructors have bachelor's degrees and the rest have
either a master's or doctorate degree.  The part-time instructors come from a
variety of employment backgrounds and have different reasons for working
part-time.  They may be high school instructors (active or retired), may come
from a household in which only one member is working full time while the other
teaches part time, may be recently graduated MS or MAT students seeking full
time employment, may be working full time elsewhere in a non-educational field,
or may be retired from a non-educational field (see
\vref{app:sec:facultyDegrees}).
%  Is it true that a quarter of our PT faculty only have BS degrees?  How did
%  this happen?  I don't see room for that in the IQs.  How did you get this
%  data?  

\subsection[Faculty diversity]{How the faculty composition reflects the diversity and cultural
competency goals of the institution.}
The mathematics SAC is deeply committed to fostering an inclusive campus
climate at each location that respects all individuals regardless of race,
color, religion, ethnicity, use of native language, national origin, sex,
marital status, height/weight ratio, disability, veteran status, age, or sexual
orientation.  Many of these human characteristics noted above are not
measurable nor necessarily discernible.  However, PCC does gather data on
gender and race/ethnicity, as detailed in \cref{reflect:tab:racialethnicmakeup}
(see also the extensive demographic data displayed in
\vref{app:sec:demographicdata}).

\begin{table}[!htb]
  \centering
  \caption{Racial/Ethnic Make-up of PCC Faculty and Students}
  \label{reflect:tab:racialethnicmakeup}
  \begin{tabular}{rrrr}
    \toprule
            &PT Faculty &   FT Faculty  & Students\\
            \midrule
    Male    & 54.1\%    & 53.2\%     & 55\% \\
    Female  & 45.9\%    & 47\%       & 45\% \\
    Asian /Pacific Islander & 7.7\% &   6.4\%    & 8\%\\
    Black or African American &   1.1\%  &  0.0\% &   6\%\\
    Hispanic/Latino &  2.2\%   & 4.3\%  &  11\%\\
    Multiracial & 1.1\%  &  0.0\%  &  3\%\\
    Native American & 0.0\%  &  0.0\%  &  1\%\\
    Unknown/International   & 12.2\% &  4.3\% &    3\%\\
    Caucasian   & 75.7\%   & 85.1\%   & 68\%\\
    \bottomrule
  \end{tabular}
\end{table}

Our SAC will continue to strive toward keeping our faculty body ethnically
diverse and culturally competent, but it is an area where improvement is
needed. In terms of hiring, there is a shortage of minorities in the Science, Technology, Engineering
and Mathematics (STEM) undergraduate and graduate programs, which makes our
recruitment of minority faculty difficult. \label{reflect:page:stem}

  \recommendation{Math chairs and deans should strongly recommend that full-time faculty
    attend workshops related to diversity and cultural competency issues.}

  \recommendation{Departments should be encouraged to provide diversity/cultural
  competency training for part-time faculty as part of their contractual
  meeting requirements.}

\recommendation{Hiring committees need to work with HR to identify
  and aggressively target mathematics graduate programs in the Northwest with
  minority students who are seeking teaching positions in community colleges.}

  \recommendation{Departments on all campuses should increase efforts to find candidates
    for the Faculty Diversity Internship Program \cite{affirmativeaction}.}

\section[Changes to instructor qualifications]{Report any changes the SAC has made to instructor qualifications since
the last review and the reason for the changes.}
In Spring 2011, prompted by the transfer of Math 20 from Developmental
Education (DE) to the Mathematics SAC, the math instructor qualifications were
changed.  Math 20 had been the only remaining mathematics course in the DE SAC.  

The transfer included transitioning three full-time DE math instructors at
Sylvania into the Math Department at Sylvania.  At this time, instructor
qualifications for math faculty were examined and changed to reflect the
inclusion of DE math faculty.  It was determined that separate qualifications
should be written for pre-college and college level courses.  These
qualifications were written so that all of the full-time DE math faculty
transitioning into the math department (as well as any new DE math faculty
hired) were qualified to teach the pre-college level courses and any new math
faculty were qualified to teach all of the math courses.  

For instance, a masters degree in mathematics education (instead of just
mathematics) was included as an optional qualification for full-time
instructors teaching pre-college level courses.  Also a masters degree in
mathematics education became an option for part-time instructors teaching MTH
211--213 (the sequence for elementary education math teachers).  Additionally,
at the request of the administration, the terms `part-time' and `full-time'
were removed from instructor qualifications in order to satisfy accreditation
requirements.  Instead of labeling what had traditionally been part-time
qualifications as `part-time,' these qualifications were labeled `Criteria for
Provisional Instructors.'

In Winter 2013, the math instructor qualifications were again changed at the 
request of the math department chairs.  The `provisional' labeling
from the last revision had required math department chairs to regularly
re-certify part-time (`provisional') instructors.  In order to avoid this
unnecessary paperwork, the SAC adopted a three-tiered qualification structure
based on full-time, part-time, and provisionally-approved part-time instructors
(mainly graduate students currently working on graduate degrees).  The
part-time (non-provisional) tier was labeled `Demonstrated Competency.'
Complete details of instructor qualifications are given in \vref{app:sec:instructorquals}.


\section[Professional development activities]{How have professional development activities of the faculty contributed to the strength of the program/discipline? If such activities have resulted in instructional or curricular changes, please describe.}

The members of the mathematics SAC, full-time and part-time alike, are very
committed to professional development.  As with members of any academic
discipline, the faculty in the math SAC pursue professional development in a
variety of manners.  Traditionally these activities have been categorized in
ways such as `membership in professional organizations' or `presentations at
conferences'.  The members of the math SAC do not in any way devalue the
engagement in such organizations or activities, and in fact a summative list of
such things can be found in \vref{app:sec:memberships}.

Nor do the members in any way diminish individual pursuit of professional
development.  In an attempt to acknowledge such pursuits, each member of the
full-time faculty was asked to submit one or two highlights of their
professional development activities over the past five years.  Those
submissions can be found in \vref{app:sec:professionaldevelop}.

It should be noted that the list of organizations and activities found in these
appendices are not exhaustive; they are merely a representative sample of
the types of professional development pursuits engaged in by members of the
math SAC.

The members of the math SAC realize that if there is going to be
institutional-level change that results in increased success and completion
rates for students enrolled in DE mathematics courses, there are going to have
to be targeted and on-going professional development activities with that goal
in mind and that all mathematics faculty, full-time and part-time, are going to
have to take advantage of those opportunities.  This is especially important
since many of our faculty members are not specialists in working with
developmental mathematics students. We look forward to working with the broader
PCC community as we pursue our common goal of increased student success and
completion, and we look forward to the college's support in providing
professional development opportunities that promote attainment of this goal.

\recommendation{The college should provide funds and other necessary resources
that allow the SAC members to engage in targeted, on-going professional
development geared toward realization of district-wide goals.  This should
include, for example, support for activities such as annual two-day workshops
focusing on goals such as universal adoption of evidence -based best
practices.}

\recommendation{Each department should create structures and policies that
promote sustained professional development.  Institutionalization of practices
such as faculty-inquiry-groups and peer-to-peer classroom visitations are
necessary components of sustained professional development.}

\recommendation{The college should continue to provide funds for activities
such as conference attendance, professional organization membership, etc. At
the same time, procedures should be put into place that allow for maximal
dissemenation of "good ideas" and maximum probability that said ideas grow into
sustained practices.}

\recommendation{Formalized procedures for mentoring new faculty, full-time and
part-time alike, should be adopted and strictly observed.  Beginning a new job
is a unique opportunity for rapid professional development, and we need to make
sure that we provide as supportive and directed an opportunity for new faculty
as possible so that the development happens in a positive and long-lasting
way.}
