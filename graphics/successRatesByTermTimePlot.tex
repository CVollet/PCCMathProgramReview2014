% arara: pdflatex
% !arara: indent: {overwrite: yes}
\documentclass{standalone}
% Caption: Success rates by year and term (Success rates decline)
% 2008-2012

\usepackage{pgfplots,pgfplotstable}
\pgfplotsset{compat=newest}

\begin{document}

\pgfplotstableread{
	Year	Fall	Winter	Spring	Summer
	2008	0.6763020184	0.6954089506	0.6728676034	0.7086156825
	2009	0.6936344969	0.7029561672	0.6449203187	0.6685520362
	2009	0.6936344969	0.7029561672	0.6449203187	0.6685520362
	2010	0.6589605735	0.6775119617	0.6449203187	0.6820288363
	2011	0.6439919445	0.6241371632	0.5989427718	0.6448675497
	2012	0.6174436761	0.6360416205	0.6098600122	0.6244411326
	}\mydata

\begin{tikzpicture}
	\begin{axis}[
			%ybar stacked,
			axis y discontinuity=crunch,
			symbolic x coords={2008, 2009, 2010, 2011, 2012},
			xtick=data,
			enlarge x limits,
			scale only axis,       
			xticklabels = {2008/09,2009/10, 2010/11, 2011/12, 2012/13},
			grid = both,
			ymin=0.56,ymax=0.75,
			yticklabel=,
			scaled ticks=false, 
			tick label style={/pgf/number format/fixed},
			legend pos= north east,
			width=\textwidth,
		]
		\addplot table[x=Year,y=Summer]{\mydata};
		\addplot table[x=Year,y=Fall]{\mydata};
		\addplot table[x=Year,y=Winter]{\mydata};
		\addplot table[x=Year,y=Spring]{\mydata};
		\legend{Summer, Fall, Winter, Spring}
	\end{axis}
\end{tikzpicture}
\end{document}
