% arara: pdflatex: {files: [MathSACpr2014]}
\chapter{Program/Discipline Overview}

\epigraph{In two years, Portland Community College will be nationally known for
progress from developmental math to college level courses and completion.}
{Chris Chairsell, Vice President of Academic and Student Affairs; Portland 
Community College Inservice, September 16, 2013}

\section{What are the educational goals or objectives of this
program/discipline.   How do these compare with national or professional
program/discipline trends or guidelines?   Have they changed since the last
review, or are they expected to change in the next five years? }

As with any undergraduate or developmental education department, the primary
goal of the faculty in the mathematics SAC can be summarized as follows: we hope
to support students' life goals by imparting the skills and cognitive
abilities necessary for continued success as they navigate their way through the
education system and into the workforce.

As evidenced in the remainder of this document, we have an active faculty who
are continually trying innovative strategies to achieve this goal.  Many of
these strategies have been targeted directly at increasing student success and
completion, such as
\begin{itemize}
  \item Accelerated Math Placement and other placement enhancement tools;
  \item study-skills focused classroom activities;
  \item experimentation with interactive homework/learning systems as well as
  \item development of said systems targeted to our students;
  \item establishment and dissemination of best practices for online accessibility.
\end{itemize}

While each of these are worthy strategies, it has become increasingly apparent
that something of greater scope needs to take place if we hope to see dramatic
changes in the success and completion rates for students taking mathematics
courses - especially those who initially place into developmental math courses.
The call for change nationwide in community colleges from access to access and
completion reinforces the Math SAC's awareness of the vital role we play in
creating an environment of student success and completion.  We were pleased to
hear Dr. Chairsell call for the creation of a college-wide culture for math
success, and we are very encouraged by the way in which this challenge has been
embraced by divisions such as student services.  We in the Math SAC also embrace
Dr. Chairsell's call for the creation of a college-wide culture for math
success and are dedicated to making the necessary changes to our mathematics
curriculum in order to maximize success and completion rates for our students.
By necessity, the most dramatic changes will need to take place in our
developmental mathematics courses.  As we restructure, we are focused on
integrating the evidence-based best practices in order to achieve the highest
rates of success and completion for our students. 

\subsection{Developmental Education (DE)}
Historically viewed as `remediation', developmental education has often
been marginalized by higher education entities.  In fact, developmental
education demonstrates a core of knowledge that is backed by extensive research.
The two largest organizations involved in developmental education research and
professional development, National Council of Developmental Education (NCDE) and
National Association of Developmental Education (NADE), define developmental
education as \emph{a comprehensive process that focuses on the intellectual,
social, and emotional growth and development of all students}.

The number of reasons students place into pre-college courses are too numerous
to list. However, a large number of students entering at the DE mathematics
level have the added burden of an intense anxiety that hinders their ability to
be successful in a mathematics course.  In combination with the academic,
social, economic, and psychological issues facing students in DE math courses,
we must approach any changes with the whole student in mind.

Over the last several years, there has been a growing sense that the traditional
algebra content, as currently taught in our DE math courses, was not meeting the
needs of many of our students. The fact that in Fall 2013 over twenty-five SAC
members joined the DE Math subcommittee  formed specifically to take a deeper
look at developmental math sequence is a strong indicator of the interest and
concern we hold.

\subsection{Science, Technology, Engineering, and Mathematics (STEM)}
Another emerging trend over the past five years has been a nationwide spotlight
on STEM education and the dire need to increase the
number of college students who ultimately obtain undergraduate degrees in STEM fields. In
fact, increasing the number of undergraduate STEM majors by 1 million over the
next decade has been formally designated as a Cross-Agency Priority (CAP) goal
by President Obama.
\footnote{\url{http://www.whitehouse.gov/blog/2012/12/18/one-decade-one-million-more-stem-graduates}}

The CAP goal proposes to focus efforts in five promising areas of opportunity:
\begin{itemize}
  \item Identifying and implementing evidence-based practices to improve STEM teaching
    and to attract students to STEM courses (see
    \cpageref{webworkposter,reflect:page:stem});
\item Providing more opportunities for students to engage in meaningful  STEM
activities through research experiences, especially in their first two years of
college;
\item Addressing the mathematics preparation gap that students face when they arrive
at college, using evidence-based practices that generate improved results;
\item Providing educational opportunities and supports for women and historically
underrepresented minorities; and 
\item Identifying and supporting innovation in higher
education.
\end{itemize}

The Math SAC realizes that for many students entering at the developmental
education level, math courses serve as a barrier for those who might otherwise
choose to pursue careers in STEM; this is well documented in studies such as
PCAST: Engage to Excel \cite{engagetoexcel}. As we work to recreate our developmental math curriculum, we are mindful of
the need to reform our courses in such a way that they no longer serve as a
barrier to the success of our students, but that they also serve as a gateway to
STEM careers for students who may have steered away from math in the past.  Most
of the goals stated as CAP \emph{areas of opportunity} include elements that
can be addressed in our courses, and we hope to create courses that support
attainment of those goals.

In doing this work, we have an eye not only toward students who (eventually)
pursue four-year STEM degrees, but we also have a focus on students enrolled in
PCC's many CTE programs.  We are committed to creating courses that support
success and completion for students enrolled in CTE programs.  Our courses must
not only promote successful completion of the math course, but they must also
impart skills that are specifically needed by the students in their CTE
courses and ultimately in their chosen careers.

\subsection{The future of DE and Undergraduate Math at PCC}
While we are still in conversation, some themes have begun to emerge.
Preliminary discussions have transpired that might lead us to
revamp our developmental education courses with an emphasis on 
\begin{itemize}
  \item evidence-based best practices;
streamline the developmental education sequence;
create developmental education sequences which support STEM education and,
ideally, promotes STEM education; 
\item integrating content into our developmental math
courses that will create a math literate populace (intelligent consumers of data
and problem solvers);
\item tracking our progress through data-analysis and assessment that ensures that
completion measures of  pass/fail rates do not mask a decrease in quality
education.
\end{itemize}

While our current focus is on DE and STEM, we are also mindful of the need to
reexamine our undergraduate level courses. The content and teaching practices we
adopt for our developmental  education courses need to be created with a clear
understanding of the potential effects those changes will have on the students
enrolled in our undergraduate level courses. Additionally, we need to ensure
that our commitment to using evidence-based best practices makes its way into
the classroom for all of our courses, not just our developmental education
courses.

We are excited by this opportunity to restructure our courses in ways that
better support student success and completion. We realize that this change
cannot be developed or implemented in isolation and we look forward to
discussing our ideas for DE restructure at the program review meeting. We also look
forward to continuing collaborative conversations with all
stakeholders including, yet not limited to, administration, CTE faculty,
advising, counseling, testing, student services, union representatives, and -
ultimately - the students themselves.

\section{Please summarize changes that have been made since the last
review.}
The mathematics department faculty is continually striving to improve our
courses.  The recommendations from the 2003--2008 Program Review (PR) \cite{mathprogramreview2003}
resulted in several changes as outlined in \vref{over:sec:changesresult}.  Some of the following 
changes that are mentioned in this section also appear at different sections of this document
as noted.
\begin{description}
  \item[Discontinue MTH 231 and MTH 232 our discrete mathematics courses] In the Spring of 2009 
    the mathematics SAC voted to discontinue offering MTH
    231 and 232, our discrete mathematics courses.  The students taking these
    courses were mostly computer science students fulfilling requirements at
    PSU.  In order for the courses to transfer, the math department coordinated
    with the PCC and PSU computer science departments with respect to
    curriculum.  For various reasons it was mutually decided that the PCC
    computer science department should run the courses that are recognized
    statewide as CS 250 and 251. 
  \item[Formed the standing subcommittee, Learning Assessment Subcommittee
    (LAS)]
     The committee was formed to address the college's assessment of the core
    outcomes.  \Cref{chap:outcomes} of this document (\cpageref{chap:outcomes}) 
    outlines the results of this committee.    
  \item[Created MTH 84] In the Fall of 2010 a pilot course was created to provide
    instruction in the use of the professional freeware publishing software
    \LaTeX.   While the emphasis of the course is creating professional
    mathematical documents, the skills learned can be used in a general context.
    One online course was run each term and we received positive responses by
    students and faculty that took the course.  Students mention using the
    program in courses other than mathematics.   In May 2011 the math SAC
    approved to make the one-credit course permanent (MTH 84) and we continue to run one
    online course every term-- see \cpageref{other:sec:mth84} for more details.
  \item[Created MTH 111H] We approved the creation of a College Algebra honors
    course in the Fall of 2010.   A description can be found on \cpageref{cur:sub:111H}.
  \item[Creation of Co-chairs] At the Cascade, Rock Creek and Sylvania campuses we
    offer between 100 to 150 class offerings per term, and
    thus we require a large adjunct faculty pool to run these courses.    The
    formula used to measure the department chair load showed that each campus
    was either close to double if not more than double compared to the next
    highest faculty-chair load for any other discipline.   Starting in the Fall
    of 2010, the department chair positions at each of the mentioned campuses
    were split into co-chair positions.   Cascade, Rock Creek and Sylvania
    campuses each have two department co-chairs.
  \item[Use of WeBWorK] To further increase student accessibility and lower costs,
    in spring 2009 Alex Jordan, brought to our attention the freeware program
    WeBWorK  partially supported by  National Science Foundation grants.    The
    software is an online homework/testing
    system that can provide \emph{immediate} feedback to the student.   Spearheaded by Alex
    Jordan faculty have been working on creating databases that fit our current
    curriculum. 
    
    It is currently being used by several faculty in courses
    offered at PCC.  The advantages to the \emph{student} are that it is free and it is
    accessible to students with disabilities.   The advantages to \emph{faculty} are that
    we can adapt it to our own curriculum and can be used for other purposes
    besides the classroom.   Ideas being proposed would allow for students to
    use it for preparation before taking placement exams.  We still are in the
    beginning phases as such a proposal would need to overcome technical
    difficulties.   Winter 2014 was the first term that we were able to
    run the program using PCC servers which allowed us to control the platform
    of this program.  Up to this point we had relied upon University of Oregon
    servers which limited the capabilities of this program. Further details 
    are given on \cpageref{other:sec:webwork}.
  \item[Social Justice workgroup] Four math faculty attended the conference
    Creating Balance in an Unjust World in San Francisco in the Winter of 2012
    and were inspired to form an ongoing collaboration of math faculty to create
    assignments and projects that have a social justice theme.  These faculty
    members share their assignments with others and encourage all math faculty
    to join them when they meet; see \cpageref{cur:sub:socialJustic} for 
    more details.
  \item[Credit hour change to MTH 243] Students brought to our attention that our
    MTH 243 course was not transferring cleanly to some institutions.  To
    address this issue, MTH 243 changed from four to five credits effective Fall
    of 2012.  An explanation for the change can be found in \vref{cur:sec:other}.
  \item[Offer ALC courses at South East] To better serve students at the South
    East center, in the Fall of 2012 the South East Center started offering
    self-paced introductory algebra courses that were previously only offered at
    Sylvania, Alternative Learning Center courses (ALC). \Vref{cur:sec:other} 
    contains further explanation of these courses. 
\end{description}
Since many of our changes are of a curricular nature we address them in
different parts of this document, notably \vref{chap:otherissues}.


\section{Were any of the changes made as a result of the last review? If so,
please describe the rationale and result.}\label{over:sec:changesresult}

In the 2003--2008 PR, and the corresponding Administrative
Response (AR) a large number of recommendations were given from the Math SAC and
the Administration.  This section will look at changes that have been made due
to those recommendations and some recommendations that are still being
addressed.

One of the recommendations from the 2003--2008 PR (\cite{mathprogramreview2003}, page 30) was to transfer MTH
20 from the Developmental Education department to the Mathematics department.
That change has taken effect starting Fall 2013.  The change helped to align
Sylvania with the rest of the campuses as to how this course was viewed.   Due
to lack of resources (at other campuses) MTH 20 was, for all practical purposes,
under the jurisdiction of the Math department.  Due to this change instructors
teaching ALC math courses asked to also be incorporated into the Math
department, housing all math courses under one legislative body.  The move was
completed as of January 2013 (see \vref{app:sec:alc}).
\fixthis{is this reference correct}

The AR gave a list of recommendations (page 3) relating to alternative methods
of moving students through the math sequence and accelerated math sequences.  In
response to this recommendation MTH 07, 08 Accelerated Math Review (See appendix
\fixthis{reference} Creation of Two Accelerated Review Courses) were created by the Math SAC.
Now that these classes are available we are hoping to offer more sections.  This
will require more advertising when students are placed into a math class.
Additionally since the ALC classes have been moved into the Math SAC the math
faculty has become more aware of these courses.  The ALC classes were once only
available at Sylvania, but now South East Center has incorporated the sequence
and other campuses are looking into it.

Page 2 of the AR asks the Math SAC to look at assessment more and take our
Course Outcomes to the next level.  Please see \vref{chap:outcomes} (of this
document) for
details, but we have made major improvements on this front and have a standing
assessment committee and action committee.  Some of our faculty members have
roles in the college wide assessment strategies.

The success rates for MTH 91 and 92 and additionally MTH 61, 62, and 63 were
mentioned on page 2 of the AR.  After looking at the success rates of MTH 91 and
92, the Math SAC no longer offers these sections.   MTH 61, 62, and 63 are still
being offered but the Math SAC continues to work on Developmental Math and we
currently have two committees looking at Math Pathways.

Page 28 of the PR (\cite{mathprogramreview2003}) recommended that an orientation to `Studying at college'
be part of the general orientation process.  Since the college has yet to make
changes in this area, Jessica Benards created study skills videos and activities
that are currently being used by math faculty in Developmental Math Classes;
further details are discussed on \cpageref{cur:sub:studyskills}.

A recommendation in the PR (\cite{mathprogramreview2003} page 31) wanted department chairs to look at math
105's low enrollment. Since then MTH 111 B and C have been merged into a single
MTH 111 class and the enrollment numbers in MTH 105 have increased.  Additionally two committees are
currently looking at math pathways from the precollege classes that might also
increase 105 numbers.  This change also led to the adoption of a new MTH 111
textbook.  Normally changing a book wouldn't merit mention in a PR, but this book
has a different philosophy and has therefore added additional changes to the
college level math sequence.

A large number of recommendations from the last PR (\cite{mathprogramreview2003} pages 32-34) are related to
Distance Learning. We currently have a DL standing committee that looks into
these matters.  See \vref{other:sec:distancelearning} (of this document) for a list of changes and concerns
that the Distance Learning Standing Committee is currently working on.

Finally, the last PR (\cite{mathprogramreview2003}, pages 30-31) made suggestions related to faculty contact with
students outside the classroom and in the learning center.  The math faculty has
continued to support the learning center over the last five years.  At Cascade
Campus the math faculty have been working with retention specialists by creating
academic interventions for students of concern.  This program shows promise and
the specialists now have an office in the math department and three staff members.

